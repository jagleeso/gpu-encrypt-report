%% bare_conf.tex
%% V1.3
%% 2007/01/11
%% by Michael Shell
%% See:
%% http://www.michaelshell.org/
%% for current contact information.
%%
%% This is a skeleton file demonstrating the use of IEEEtran.cls
%% (requires IEEEtran.cls version 1.7 or later) with an IEEE conference paper.
%%
%% Support sites:
%% http://www.michaelshell.org/tex/ieeetran/
%% http://www.ctan.org/tex-archive/macros/latex/contrib/IEEEtran/
%% and
%% http://www.ieee.org/

%%*************************************************************************
%% Legal Notice:
%% This code is offered as-is without any warranty either expressed or
%% implied; without even the implied warranty of MERCHANTABILITY or
%% FITNESS FOR A PARTICULAR PURPOSE! 
%% User assumes all risk.
%% In no event shall IEEE or any contributor to this code be liable for
%% any damages or losses, including, but not limited to, incidental,
%% consequential, or any other damages, resulting from the use or misuse
%% of any information contained here.
%%
%% All comments are the opinions of their respective authors and are not
%% necessarily endorsed by the IEEE.
%%
%% This work is distributed under the LaTeX Project Public License (LPPL)
%% ( http://www.latex-project.org/ ) version 1.3, and may be freely used,
%% distributed and modified. A copy of the LPPL, version 1.3, is included
%% in the base LaTeX documentation of all distributions of LaTeX released
%% 2003/12/01 or later.
%% Retain all contribution notices and credits.
%% ** Modified files should be clearly indicated as such, including  **
%% ** renaming them and changing author support contact information. **
%%
%% File list of work: IEEEtran.cls, IEEEtran_HOWTO.pdf, bare_adv.tex,
%%                    bare_conf.tex, bare_jrnl.tex, bare_jrnl_compsoc.tex
%%*************************************************************************

% *** Authors should verify (and, if needed, correct) their LaTeX system  ***
% *** with the testflow diagnostic prior to trusting their LaTeX platform ***
% *** with production work. IEEE's font choices can trigger bugs that do  ***
% *** not appear when using other class files.                            ***
% The testflow support page is at:
% http://www.michaelshell.org/tex/testflow/



% Note that the a4paper option is mainly intended so that authors in
% countries using A4 can easily print to A4 and see how their papers will
% look in print - the typesetting of the document will not typically be
% affected with changes in paper size (but the bottom and side margins will).
% Use the testflow package mentioned above to verify correct handling of
% both paper sizes by the user's LaTeX system.
%
% Also note that the "draftcls" or "draftclsnofoot", not "draft", option
% should be used if it is desired that the figures are to be displayed in
% draft mode.
%
\documentclass[conference,10pt]{IEEEtran}
% Add the compsoc option for Computer Society conferences.
%
% If IEEEtran.cls has not been installed into the LaTeX system files,
% manually specify the path to it like:
% \documentclass[conference]{../sty/IEEEtran}


% Custom packages.
\usepackage{filecontents}
\usepackage{csvsimple}
\usepackage{url}



% Some very useful LaTeX packages include:
% (uncomment the ones you want to load)


% *** MISC UTILITY PACKAGES ***
%
%\usepackage{ifpdf}
% Heiko Oberdiek's ifpdf.sty is very useful if you need conditional
% compilation based on whether the output is pdf or dvi.
% usage:
% \ifpdf
%   % pdf code
% \else
%   % dvi code
% \fi
% The latest version of ifpdf.sty can be obtained from:
% http://www.ctan.org/tex-archive/macros/latex/contrib/oberdiek/
% Also, note that IEEEtran.cls V1.7 and later provides a builtin
% \ifCLASSINFOpdf conditional that works the same way.
% When switching from latex to pdflatex and vice-versa, the compiler may
% have to be run twice to clear warning/error messages.






% *** CITATION PACKAGES ***
%
%\usepackage{cite}
% cite.sty was written by Donald Arseneau
% V1.6 and later of IEEEtran pre-defines the format of the cite.sty package
% \cite{} output to follow that of IEEE. Loading the cite package will
% result in citation numbers being automatically sorted and properly
% "compressed/ranged". e.g., [1], [9], [2], [7], [5], [6] without using
% cite.sty will become [1], [2], [5]--[7], [9] using cite.sty. cite.sty's
% \cite will automatically add leading space, if needed. Use cite.sty's
% noadjust option (cite.sty V3.8 and later) if you want to turn this off.
% cite.sty is already installed on most LaTeX systems. Be sure and use
% version 4.0 (2003-05-27) and later if using hyperref.sty. cite.sty does
% not currently provide for hyperlinked citations.
% The latest version can be obtained at:
% http://www.ctan.org/tex-archive/macros/latex/contrib/cite/
% The documentation is contained in the cite.sty file itself.
\usepackage{cite}




% \usepackage{graphicx}
% *** GRAPHICS RELATED PACKAGES ***
%
\ifCLASSINFOpdf
  % \usepackage[pdftex]{graphicx}
  % I don't know... trying to solve bounding box \usepackage{auto-pst-pdf}
  % \usepackage{epstopdf}
  % declare the path(s) where your graphic files are
  % \graphicspath{{../pdf/}{../jpeg/}}
  % and their extensions so you won't have to specify these with
  % every instance of \includegraphics
  % \DeclareGraphicsExtensions{.pdf,.jpeg,.png}
\else
  % or other class option (dvipsone, dvipdf, if not using dvips). graphicx
  % will default to the driver specified in the system graphics.cfg if no
  % driver is specified.
  \usepackage[dvips]{graphicx}
  % declare the path(s) where your graphic files are
  % \graphicspath{{../eps/}}
  % and their extensions so you won't have to specify these with
  % every instance of \includegraphics
  % \DeclareGraphicsExtensions{.eps}
\fi
% graphicx was written by David Carlisle and Sebastian Rahtz. It is
% required if you want graphics, photos, etc. graphicx.sty is already
% installed on most LaTeX systems. The latest version and documentation can
% be obtained at: http://www.ctan.org/tex-archive/macros/latex/required/graphics/
% Another good source of documentation is "Using Imported Graphics in
% LaTeX2e" by Keith Reckdahl which can be found as epslatex.ps or
% epslatex.pdf at: http://www.ctan.org/tex-archive/info/
%
% latex, and pdflatex in dvi mode, support graphics in encapsulated
% postscript (.eps) format. pdflatex in pdf mode supports graphics
% in .pdf, .jpeg, .png and .mps (metapost) formats. Users should ensure
% that all non-photo figures use a vector format (.eps, .pdf, .mps) and
% not a bitmapped formats (.jpeg, .png). IEEE frowns on bitmapped formats
% which can result in "jaggedy"/blurry rendering of lines and letters as
% well as large increases in file sizes.
%
% You can find documentation about the pdfTeX application at:
% http://www.tug.org/applications/pdftex





% *** MATH PACKAGES ***
%
%\usepackage[cmex10]{amsmath}
% A popular package from the American Mathematical Society that provides
% many useful and powerful commands for dealing with mathematics. If using
% it, be sure to load this package with the cmex10 option to ensure that
% only type 1 fonts will utilized at all point sizes. Without this option,
% it is possible that some math symbols, particularly those within
% footnotes, will be rendered in bitmap form which will result in a
% document that can not be IEEE Xplore compliant!
%
% Also, note that the amsmath package sets \interdisplaylinepenalty to 10000
% thus preventing page breaks from occurring within multiline equations. Use:
%\interdisplaylinepenalty=2500
% after loading amsmath to restore such page breaks as IEEEtran.cls normally
% does. amsmath.sty is already installed on most LaTeX systems. The latest
% version and documentation can be obtained at:
% http://www.ctan.org/tex-archive/macros/latex/required/amslatex/math/





% *** SPECIALIZED LIST PACKAGES ***
%
%\usepackage{algorithmic}
% algorithmic.sty was written by Peter Williams and Rogerio Brito.
% This package provides an algorithmic environment fo describing algorithms.
% You can use the algorithmic environment in-text or within a figure
% environment to provide for a floating algorithm. Do NOT use the algorithm
% floating environment provided by algorithm.sty (by the same authors) or
% algorithm2e.sty (by Christophe Fiorio) as IEEE does not use dedicated
% algorithm float types and packages that provide these will not provide
% correct IEEE style captions. The latest version and documentation of
% algorithmic.sty can be obtained at:
% http://www.ctan.org/tex-archive/macros/latex/contrib/algorithms/
% There is also a support site at:
% http://algorithms.berlios.de/index.html
% Also of interest may be the (relatively newer and more customizable)
% algorithmicx.sty package by Szasz Janos:
% http://www.ctan.org/tex-archive/macros/latex/contrib/algorithmicx/




% *** ALIGNMENT PACKAGES ***
%
%\usepackage{array}
% Frank Mittelbach's and David Carlisle's array.sty patches and improves
% the standard LaTeX2e array and tabular environments to provide better
% appearance and additional user controls. As the default LaTeX2e table
% generation code is lacking to the point of almost being broken with
% respect to the quality of the end results, all users are strongly
% advised to use an enhanced (at the very least that provided by array.sty)
% set of table tools. array.sty is already installed on most systems. The
% latest version and documentation can be obtained at:
% http://www.ctan.org/tex-archive/macros/latex/required/tools/


%\usepackage{mdwmath}
%\usepackage{mdwtab}
% Also highly recommended is Mark Wooding's extremely powerful MDW tools,
% especially mdwmath.sty and mdwtab.sty which are used to format equations
% and tables, respectively. The MDWtools set is already installed on most
% LaTeX systems. The lastest version and documentation is available at:
% http://www.ctan.org/tex-archive/macros/latex/contrib/mdwtools/


% IEEEtran contains the IEEEeqnarray family of commands that can be used to
% generate multiline equations as well as matrices, tables, etc., of high
% quality.


%\usepackage{eqparbox}
% Also of notable interest is Scott Pakin's eqparbox package for creating
% (automatically sized) equal width boxes - aka "natural width parboxes".
% Available at:
% http://www.ctan.org/tex-archive/macros/latex/contrib/eqparbox/





% *** SUBFIGURE PACKAGES ***
\usepackage[tight,footnotesize]{subfigure}
% subfigure.sty was written by Steven Douglas Cochran. This package makes it
% easy to put subfigures in your figures. e.g., "Figure 1a and 1b". For IEEE
% work, it is a good idea to load it with the tight package option to reduce
% the amount of white space around the subfigures. subfigure.sty is already
% installed on most LaTeX systems. The latest version and documentation can
% be obtained at:
% http://www.ctan.org/tex-archive/obsolete/macros/latex/contrib/subfigure/
% subfigure.sty has been superceeded by subfig.sty.



%\usepackage[caption=false]{caption}
%\usepackage[font=footnotesize]{subfig}
% subfig.sty, also written by Steven Douglas Cochran, is the modern
% replacement for subfigure.sty. However, subfig.sty requires and
% automatically loads Axel Sommerfeldt's caption.sty which will override
% IEEEtran.cls handling of captions and this will result in nonIEEE style
% figure/table captions. To prevent this problem, be sure and preload
% caption.sty with its "caption=false" package option. This is will preserve
% IEEEtran.cls handing of captions. Version 1.3 (2005/06/28) and later (recommended due to many 
% improvements over 1.2) of subfig.sty supports
% the caption=false option directly:
%\usepackage[caption=false,font=footnotesize]{subfig}
%
% The latest version and documentation can be obtained at:
% http://www.ctan.org/tex-archive/macros/latex/contrib/subfig/
% The latest version and documentation of caption.sty can be obtained at:
% http://www.ctan.org/tex-archive/macros/latex/contrib/caption/




% *** FLOAT PACKAGES ***
%
\usepackage{fixltx2e}
% fixltx2e, the successor to the earlier fix2col.sty, was written by
% Frank Mittelbach and David Carlisle. This package corrects a few problems
% in the LaTeX2e kernel, the most notable of which is that in current
% LaTeX2e releases, the ordering of single and double column floats is not
% guaranteed to be preserved. Thus, an unpatched LaTeX2e can allow a
% single column figure to be placed prior to an earlier double column
% figure. The latest version and documentation can be found at:
% http://www.ctan.org/tex-archive/macros/latex/base/



%\usepackage{stfloats}
% stfloats.sty was written by Sigitas Tolusis. This package gives LaTeX2e
% the ability to do double column floats at the bottom of the page as well
% as the top. (e.g., "\begin{figure*}[!b]" is not normally possible in
% LaTeX2e). It also provides a command:
%\fnbelowfloat
% to enable the placement of footnotes below bottom floats (the standard
% LaTeX2e kernel puts them above bottom floats). This is an invasive package
% which rewrites many portions of the LaTeX2e float routines. It may not work
% with other packages that modify the LaTeX2e float routines. The latest
% version and documentation can be obtained at:
% http://www.ctan.org/tex-archive/macros/latex/contrib/sttools/
% Documentation is contained in the stfloats.sty comments as well as in the
% presfull.pdf file. Do not use the stfloats baselinefloat ability as IEEE
% does not allow \baselineskip to stretch. Authors submitting work to the
% IEEE should note that IEEE rarely uses double column equations and
% that authors should try to avoid such use. Do not be tempted to use the
% cuted.sty or midfloat.sty packages (also by Sigitas Tolusis) as IEEE does
% not format its papers in such ways.





% *** PDF, URL AND HYPERLINK PACKAGES ***
%
%\usepackage{url}
% url.sty was written by Donald Arseneau. It provides better support for
% handling and breaking URLs. url.sty is already installed on most LaTeX
% systems. The latest version can be obtained at:
% http://www.ctan.org/tex-archive/macros/latex/contrib/misc/
% Read the url.sty source comments for usage information. Basically,
% \url{my_url_here}.





% *** Do not adjust lengths that control margins, column widths, etc. ***
% *** Do not use packages that alter fonts (such as pslatex).         ***
% There should be no need to do such things with IEEEtran.cls V1.6 and later.
% (Unless specifically asked to do so by the journal or conference you plan
% to submit to, of course. )


% correct bad hyphenation here
\hyphenation{op-tical net-works semi-conduc-tor}


\begin{document}
%
% paper title
% can use linebreaks \\ within to get better formatting as desired
\title{Bare Demo of IEEEtran.cls for Conferences}


% author names and affiliations
% use a multiple column layout for up to three different
% affiliations
\author{\IEEEauthorblockN{Michael Shell}
\IEEEauthorblockA{School of Electrical and\\Computer Engineering\\
Georgia Institute of Technology\\
Atlanta, Georgia 30332--0250\\
Email: http://www.michaelshell.org/contact.html}
\and
\IEEEauthorblockN{Homer Simpson}
\IEEEauthorblockA{Twentieth Century Fox\\
Springfield, USA\\
Email: homer@thesimpsons.com}
\and
\IEEEauthorblockN{James Kirk\\ and Montgomery Scott}
\IEEEauthorblockA{Starfleet Academy\\
San Francisco, California 96678-2391\\
Telephone: (800) 555--1212\\
Fax: (888) 555--1212}}

% conference papers do not typically use \thanks and this command
% is locked out in conference mode. If really needed, such as for
% the acknowledgment of grants, issue a \IEEEoverridecommandlockouts
% after \documentclass

% for over three affiliations, or if they all won't fit within the width
% of the page, use this alternative format:
% 
%\author{\IEEEauthorblockN{Michael Shell\IEEEauthorrefmark{1},
%Homer Simpson\IEEEauthorrefmark{2},
%James Kirk\IEEEauthorrefmark{3}, Montgomery Scott\IEEEauthorrefmark{3} and
%Eldon Tyrell\IEEEauthorrefmark{4}}
%\IEEEauthorblockA{\IEEEauthorrefmark{1}School of Electrical and Computer Engineering\\
%Georgia Institute of Technology,
%Atlanta, Georgia 30332--0250\\ Email: see http://www.michaelshell.org/contact.html}
%\IEEEauthorblockA{\IEEEauthorrefmark{2}Twentieth Century Fox, Springfield, USA\\
%Email: homer@thesimpsons.com}
%\IEEEauthorblockA{\IEEEauthorrefmark{3}Starfleet Academy, San Francisco, California 96678-2391\\
%Telephone: (800) 555--1212, Fax: (888) 555--1212}
%\IEEEauthorblockA{\IEEEauthorrefmark{4}Tyrell Inc., 123 Replicant Street, Los Angeles, California 
%90210--4321}}




% use for special paper notices
%\IEEEspecialpapernotice{(Invited Paper)}




% make the title area
\maketitle


\begin{abstract}
%\boldmath
The abstract goes here.
\end{abstract}
% IEEEtran.cls defaults to using nonbold math in the Abstract.
% This preserves the distinction between vectors and scalars. However,
% if the conference you are submitting to favors bold math in the abstract,
% then you can use LaTeX's standard command \boldmath at the very start
% of the abstract to achieve this. Many IEEE journals/conferences frown on
% math in the abstract anyway.

% no keywords




% For peer review papers, you can put extra information on the cover
% page as needed:
% \ifCLASSOPTIONpeerreview
% \begin{center} \bfseries EDICS Category: 3-BBND \end{center}
% \fi
%
% For peerreview papers, this IEEEtran command inserts a page break and
% creates the second title. It will be ignored for other modes.
\IEEEpeerreviewmaketitle



\section{Introduction}
Madeup \cite{hoard}.
% no \IEEEPARstart
This demo file is intended to serve as a ``starter file''
for IEEE conference papers produced under \LaTeX\ using
IEEEtran.cls version 1.7 and later.
% You must have at least 2 lines in the paragraph with the drop letter
% (should never be an issue)
I wish you the best of success.

\hfill mds
 
\hfill January 11, 2007

\subsection{Subsection Heading Here}
Subsection text here.


\subsubsection{Subsubsection Heading Here}
Subsubsection text here.


% An example of a floating figure using the graphicx package.
% Note that \label must occur AFTER (or within) \caption.
% For figures, \caption should occur after the \includegraphics.
% Note that IEEEtran v1.7 and later has special internal code that
% is designed to preserve the operation of \label within \caption
% even when the captionsoff option is in effect. However, because
% of issues like this, it may be the safest practice to put all your
% \label just after \caption rather than within \caption{}.
%
% Reminder: the "draftcls" or "draftclsnofoot", not "draft", class
% option should be used if it is desired that the figures are to be
% displayed while in draft mode.
%
%\begin{figure}[!t]
%\centering
%\includegraphics[width=2.5in]{myfigure}
% where an .eps filename suffix will be assumed under latex, and a .pdf suffix will be assumed for 
% pdflatex; or what has been declared
% via \DeclareGraphicsExtensions.
%\caption{Simulation Results}
%\label{fig_sim}
%\end{figure}

% Some.
% \begin{figure}[h]
% \centering
% \includegraphics[width=0.4\linewidth,angle=-90]{benchmarks/cache-scratch/cache-scratch.ps}
% \caption{Performance and scalability of allocators on threadtest}
% \label{fig:threadtest}
% \end{figure}

The OpenCL programming model defines work items to be an executing instance of an OpenCL kernel 
(assigned a local id), and work groups to be equally sized groups of work items (assigned a global 
id). The OpenCL model ensures that work items within a work group will execute concurrently, but 
provides no guarantees whether work groups will execute concurrently or serially. When you execute 
an OpenCL kernel, you specify the number of work groups, and the number of work items within each 
work group. All work items within a group each get a local id that is unique within that group as 
well as a global id that is unique across all instances of the executing kernel. The work items then 
make use of their local and global ids to appropriately divide up the work, and can make use of 
synchronization constructs like a work group barrier to synchronize their execution (but only for 
work items within the same group) \cite{opencl_guide}.

An existing implementation of an OpenCL kernel implementing the AES encryption algorithm was 
obtained from a public source code repository \cite{opencl_impl}.  The original implementation of 
the OpenCL encrypt kernel is implemented using a data parallel programming model, whereby a program 
is parallelized by performing the same computation on different pieces of data \cite{opencl_guide}. 
In particular, it only makes use of the global id of a work group to index itself into the input 
array at a particular position (e.g. index 0 of the input array), with one work group instance being 
responsible for encrypting an entry of the input array. The kernel also makes use of the vector 
instruction set of the GPU to increase the number of bytes that a single "entry" in the input array 
corresponds to. In particular, in addition to traditional analogs to C data types like uint (a 
32-bit integer for a total of 4 bytes), there are vector data types like uint4 (4 32-bit integers, 
for total of 16 bytes). Using these vector types allows one to execute native vector operations for 
the target platform \cite{opencl_guide}.  In particular, a single work item can operate on 16 bytes 
at once, whereas a traditional CPU would operate on 4 bytes at once. Thus, given an input array of 
$N$ bytes, we only need $N/16$ work groups of 1 work item each to encrypt an input array. Hence, in 
this implementation, the local id is ignored, since a work group only consists of a single work 
item.

The target device used for conducting experiments was a MotoX phone (table \ref{table:specs}).  We 
initially intended to use the Nexus 4 phone, but this was abandoned due to apparent instabilities 
triggered by the use of OpenCL on the device. 

Preliminary experiments determined that the full parallelism offered by the GPU was not being 
exploited using the original implementation.  Hence, modifications were made to the original 
implementation in order to explore the different levels of parallelism offered by varying the input 
parameters to the kernel such as the number of work groups and the work group size (which were 
previously strictly $N/16$ and 1 respectively).  The following sections describe the modifications 
that were performed to the original implementation and investigates their effect on the throughput 
of encryption.

\subsubsection{Even partitioning by work items}
\label{subsec:impl_partition}

We modified the OpenCL AES encrypt kernel such that the input array is split up evenly amongst the 
work items across all work groups (whereas previously they only encrypted 16 bytes). In particular, 
all OpenCL instances get some multiple of 16 bytes on which to operate, all of which are equally 
sized except one OpenCL instance which will be allocated whatever remains of the input array (i.e. 
given N = 128 bytes, and 3 work items, then 2 instances will operate on 48 bytes and 1 instance will 
operate on 32 bytes). Timing experiments were performed on a 128 MB input array, with the total 
number of work groups being varied from 1 up to 16, and the work group size being kept at 1 (refer 
to figure \ref{fig:num_work_groups}).

\begin{figure}[!t]
\centering
\includegraphics[width=2.5in]{../final/motox/4.2/sample_opencl_aes_global_worksize.128MB.16_max_global_worksize.again.report.eps}
\caption{OpenCL performance over a varying number of work groups}
\label{fig:num_work_groups}
\end{figure}

Going from of 1 up to 4 work groups, we see a linear speedup in encryption throughput (going from 
work groups of 1 to 2 we see a 2.37 MB/second speedup, 2 to 3 we see 2.36 MB/second, and 3 to 4 we 
see 2.36 MB/second). These linear speedups going from 1 up to 4 work groups are consistent with the 
device specifications in table \ref{table:specs} which state that the GPU has 4 computing units 
(hence, it is using an additional computing unit for each additional work group).

When we reach 5 work groups, we see the worst decrease in encryption throughput (3.54 MB/second). 
This is most likely due to the OpenCL runtime scheduling 4 simultaneous OpenCL kernel instances (4 
work groups each consisting of a single work item) that operate on $1/5$ of the input array, with a 
single OpenCL instance running on the remaining $1/5$ only after the first 4 complete (thereby 
underutilizing the parallelism of the GPU).

The throughput degrades the most for numbers of work groups that that are 1 modulo 4 since given 
that instances will tend to complete in roughly the same time (since they have the same data size 
and instructions executed), there will tend to be a point at which only one GPU core will be 
utilized, thus reducing parallelism (similarly for 2 modulo 4, and 3 modulo 4). As we increase the 
number of work groups, the degradation is less since the point at which we aren't maximally using 
all 4 GPU cores operates over less of the input array.  This phenomenon has been described in 
\cite{gpu_opt} as the tail effect.

Hence, from figure \ref{fig:num_work_groups}, the main conclusion to be drawn is that work groups 
are being scheduled on compute devices, but only one work group at a time is being scheduled 
(otherwise the peaks would reach about 10 MB/second as the number of work groups increases).

Next, we investigated the affect of increasing the number of work items used on a single compute 
device.  In particular, we began by querying the OpenCL runtime for the maximum available work group 
size, which was found to be 256.  However, this is only a theoretical maximum, and the actual 
maximum enforced by OpenCL is dependant on the kernel being executed and its resource requirements 
\cite{opencl_guide}; in our case, the maximum available was 80.  Timing experiments were performed 
on a 32 MB input array, with the total number of work items being varied from 1 up to 80, and the 
number of work groups being kept at 1 (refer to figure \ref{fig:work_group_size}).

\begin{figure}[!t]
\centering
\includegraphics[width=2.5in]{../final/motox/4.2/sample_opencl_aes_work_group_size.32MB.1_work_groups.again.report.eps}
\caption{OpenCL performance over varying work group size}
\label{fig:work_group_size}
\end{figure}

From the graph, we can see the encryption throughput gradually increases from 0 up to 20 work items.  
However, after 20 work items, we observe some degradation, followed by the throughput leveling off 
at around 18 MB/second in throughput.  This decrease in throughput for an increase in processing 
power usage is an unexpected result and warranted further investigation.  In particular, 





I used the results from the graph in my previous email to guide me in using an appropriate number of 
work groups (4, equal to the number of GPU cores). I then varied the size of each work group from 
the minimum (1 processing element) up to the maximum (80 processing elements).

Refer to attached graph.

Summary:
From the graph, we can see that the maximum throughput achieved is 57719.41 bytes/ms (~ 55.05 MB/s) 
when using a global work-size of 4, with a local worksize (a.k.a work group size) of 50. Mind you, 
the maximum is nearly reached by the time we are using a local worksize of 20. I'm guessing this is 
related to caching behaviour, but I am not yet familiar with cache layout in GPUs.




\begin{figure}[!t]
\centering
\includegraphics[width=2.5in]{../final/motox/4.2/opencl_sizes_vs_cpu_sizes.opencl_4G_19L.cpu_1thread.report.eps}
\caption{OpenCL GPU vs OpenSSL performance in AES encryption}
\label{fig:opencl_vs_cpu}
\end{figure}

We are able to outperform a single CPU by about 7MB/s. However, we certainly aren't beating the CPU 
handily.

From here, I am investigating literature for experimental results and implementation details of AES 
encryption on the GPU.  In particular, I need to understand more about the OpenCL memory model to 
better understand what techniques published in literature are effective and why.


Hello everyone,

Stefan, I wasn't able to get to trying zeroing pages in place of encryption yet.  However, I intend 
to do that next, since I am wanting to know (and I think you are too) what the memory bandwidth is 
for this thing.

What I did do (in addition to the plot I showed in my last email), was take my modified AES (that 
splits the data to be encrypted into equal sized chunks), and investigate 2 parameters:

    work group size: the number processing elements to use on a GPU core
    number of work groups: the number GPU cores that are utilized

In my previous graph I already looked at varying the work group size (I referred to it as local 
worksize, but I'll stop using that term), but that was when using 4 GPU cores.  So, the following 
graph investigates work group size for 1 GPU core.

I observed similar results to my previous graph that varied work group size for 4 GPU cores: max 
throughput is achieved at around a work group size of 19-20, with no gains after that.

Next, to get a basis of comparison to the CPU, I implemented a benchmark of OpenSSL encryption.  I 
used the fact that we achieved optimal encryption around a work group size of 19, and the number of 
work groups set to 4.  I then varied the input size from 16 bytes ... 512 MB (incrementing by powers 
of 2).


\begin{figure}[!t]
\centering
\includegraphics[width=2.5in]{../final/motox/4.2/sample_opencl_aes_entries.64MB.4_work_groups.5000_max_entries.both.eps}
\caption{OpenCL performance over for strided vs coalesced accesses within a work group}
\label{fig:global_worksize}
\end{figure}

Hey everyone,

I spent this week trying to investigate the reason for the level off in performance observed when 
using 1 GPU core and varying the number of processing elements used on that core (the first graph in 
the last email).  In particular, I've been researching general performance optimizations for GPU 
programming. 

The first thing I tried doing was varying the number of entries (where an entry is a contiguous 16 
bytes of input) encrypted by each work item.  This is to test the hypothesis that by my 
implementation splitting the input to be encrypted equally amongst work items, increasing the number 
of entries that each work item encrypts would introduce increasingly larger stride distances between 
memory accesses of work items in a work group.  This is an undesirable property if the GPU is 
reading in memory in chunks, since increasingly distant memory accesses will be more likely to 
require a separate chunk being read (i.e. no locality).  In summary, memory accesses are said to be 
coalesced if multiple threads access contiguous memory locations (whereas non-contiguous accesses 
are referred to as strided accesses).

I set the work group size to the maximum available (80 processing elements per GPU core), with all 4 
GPU cores are being utilized.  Hence, I am maximizing the parallelism of the GPU hardware for this 
experiment.

From the graph, we see that we generally achieve the best encryption throughput for small number of 
entries (< 500), and it tends to get worse at the peaks for increasingly large sizes per work item, 
which I believe is due to increasingly large stride distances within a work group. I believe the 
fluctuations observed in the graph are due to the GPU being underutilized near the end of the 
computation.  In particular, in order to use a work group size of 80, the OpenCL runtime forces me 
to make each work group that size.  If the number of processing elements needed doesn't evenly 
divide the input, this will cause the final scheduled work group to underutilize the GPU (and it 
becomes increasingly bad if the processing element is responsible for a larger encryption size).

Given this result, I then tried to see if I could modify my kernel not to introduce strided memory 
accesses within a work group.  In particular, the implementation generating the graph above has work 
items within a single work group accessing entries like:

Work item 0: A[i], A[i + 1], ... A[i + entries - 1]
Work item 1: A[i + entries], A[i + entries + 1], ... A[i + 2*entries - 1]

(e.g. A[i] and A[i + entries] are read concurrently and are strided accesses)

I modified my implementation to make it so accessing entries within a single work group looks like 
(given that G is the number of work groups):

Work item 0: A[i], A[i + entries], ... A[i + entries*(G - 1)]
Work item 1: A[i + 1] A[i + 1 + entries], ..., A[i + 1 + entries*(G - 1)]

(e.g. A[i] and A[i+1] are read concurrently and are coalesced memory accesses)

Refer to the graph below for the results:
We observe nearly identical throughput behaviour (in particular, when it decreases/increases) as 
before.  However, we get slightly less throughput, which I believe is due to the increased 
operations I am performing for entry index calculations.  If memory coalescing is in fact the reason 
for poor performance, then this graph indicates that memory accesses are still not coalesced.  I am 
confident that memory operations are coalesced within a work group, but it occurred to me that 
memory accesses are not being coalesced between work groups.  From reviewing research, it seemed as 
though the focus was coalesced access within work groups (presumably because we are guaranteed that 
these accesses will happen simultaneously), so it is not clear whether it matters between work 
groups (whose accesses may not happen simultaneously).

% Note that IEEE typically puts floats only at the top, even when this
% results in a large percentage of a column being occupied by floats.


% An example of a double column floating figure using two subfigures.
% (The subfig.sty package must be loaded for this to work.)
% The subfigure \label commands are set within each subfloat command, the
% \label for the overall figure must come after \caption.
% \hfil must be used as a separator to get equal spacing.
% The subfigure.sty package works much the same way, except \subfigure is
% used instead of \subfloat.
%
%\begin{figure*}[!t]
%\centerline{\subfloat[Case I]\includegraphics[width=2.5in]{subfigcase1}%
%\label{fig_first_case}}
%\hfil
%\subfloat[Case II]{\includegraphics[width=2.5in]{subfigcase2}%
%\label{fig_second_case}}}
%\caption{Simulation results}
%\label{fig_sim}
%\end{figure*}
%
% Note that often IEEE papers with subfigures do not employ subfigure
% captions (using the optional argument to \subfloat), but instead will
% reference/describe all of them (a), (b), etc., within the main caption.


% An example of a floating table. Note that, for IEEE style tables, the \caption command should come 
% BEFORE the table. Table text will default to
% \footnotesize as IEEE normally uses this smaller font for tables.
% The \label must come after \caption as always.
%
%\begin{table}[!t]
%% increase table row spacing, adjust to taste
%\renewcommand{\arraystretch}{1.3}
% if using array.sty, it might be a good idea to tweak the value of
% \extrarowheight as needed to properly center the text within the cells
%\caption{An Example of a Table}
%\label{table_example}
%\centering
%% Some packages, such as MDW tools, offer better commands for making tables
%% than the plain LaTeX2e tabular which is used here.
%\begin{tabular}{|c||c|}
%\hline
%One & Two\\
%\hline
%Three & Four\\
%\hline
%\end{tabular}
%\end{table}

\begin{table}[]
\centering
\caption{MotoX device specifications}
\label{table:specs}
\csvautotabular{resource/motox_specs.txt}
\end{table}



% Note that IEEE does not put floats in the very first column - or typically
% anywhere on the first page for that matter. Also, in-text middle ("here")
% positioning is not used. Most IEEE journals/conferences use top floats
% exclusively. Note that, LaTeX2e, unlike IEEE journals/conferences, places
% footnotes above bottom floats. This can be corrected via the \fnbelowfloat
% command of the stfloats package.



\section{Conclusion}
The conclusion goes here.




% conference papers do not normally have an appendix


% use section* for acknowledgement
\section*{Acknowledgment}


The authors would like to thank...





% trigger a \newpage just before the given reference
% number - used to balance the columns on the last page
% adjust value as needed - may need to be readjusted if
% the document is modified later
%\IEEEtriggeratref{8}
% The "triggered" command can be changed if desired:
%\IEEEtriggercmd{\enlargethispage{-5in}}

% references section

% can use a bibliography generated by BibTeX as a .bbl file
% BibTeX documentation can be easily obtained at:
% http://www.ctan.org/tex-archive/biblio/bibtex/contrib/doc/
% The IEEEtran BibTeX style support page is at:
% http://www.michaelshell.org/tex/ieeetran/bibtex/
%\bibliographystyle{IEEEtran}
% argument is your BibTeX string definitions and bibliography database(s)
%\bibliography{IEEEabrv,../bib/paper}
%
% <OR> manually copy in the resultant .bbl file
% set second argument of \begin to the number of references
% (used to reserve space for the reference number labels box)
% \begin{thebibliography}{1}
% 
% \bibitem{IEEEhowto:kopka}
% H.~Kopka and P.~W. Daly, \emph{A Guide to \LaTeX}, 3rd~ed.\hskip 1em plus
%   0.5em minus 0.4em\relax Harlow, England: Addison-Wesley, 1999.
% 
% \end{thebibliography}

% \begin{filecontents*}{bibliography.bib}
% 
%     @article{hoard,
%         title={Hoard: A scalable memory allocator for multithreaded applications},
%         author={Berger, Emery D and McKinley, Kathryn S and Blumofe, Robert D and Wilson, Paul R},
%         journal={ACM SIGPLAN Notices},
%         volume={35},
%         number={11},
%         pages={117--128},
%         year={2000},
%         publisher={ACM}
%     },
% 
%     @misc{seq_mem_alloc,
%         title = {{Sequential Memory Allocators}},
%         howpublished = 
%         {\url{http://www.cdf.toronto.edu/~csc469h/fall/assignments/a2/malloc_notes.pdf}},
%         note = {Accessed: 2013-11-11}
%     },
% 
% 
% 
% \end{filecontents*}

    % @misc{why,
    %     title = {{Sequential Memory Allocators}},
    %     howpublished = 
%     {\url{http://www.cdf.toronto.edu/~csc469h/fall/assignments/a2/malloc_notes.pdf}},
    %     note = {Accessed: 2013-11-11}
    % },

\begin{filecontents*}{bibliography.bib}

    @misc{specs,
        title = {{PDAdb.net} Qualcomm Snapdragon S4 Pro MSM8960DT Multi-core Application Processor 
        with Modem Datasheet},
        howpublished = 
        {\url{http://www.pdadb.net/index.php?m=cpu&id=a8960dt&c=qualcomm_snapdragon_s4_pro_msm8960dt}},
        note = {Accessed: 2013-12-04}
    },

    @misc{gpu_opt,
        title = {{NVIDIA} GPU Performance Analysis and Optimization},
        howpublished = 
        {\url{http://on-demand.gputechconf.com/gtc/2012/presentations/S0514-GTC2012-GPU-Performance-Analysis.pdf}},
        note = {Accessed: 2013-12-04}
    },

    @misc{opencl_impl,
        title = {{GitHub} softboysxp/OpenCL-AES},
        howpublished = {\url{https://github.com/softboysxp/OpenCL-AES}},
        note = {Accessed: 2013-12-04}
    },

    @article{hoard,
        title={Hoard: A scalable memory allocator for multithreaded applications},
        author={Berger, Emery D and McKinley, Kathryn S and Blumofe, Robert D and Wilson, Paul R},
        journal={ACM SIGPLAN Notices},
        volume={35},
        number={11},
        pages={117--128},
        year={2000},
        publisher={ACM}
    },

    @book{opencl_guide,
        title={OpenCL programming guide},
        author={Munshi, Aaftab and Gaster, Benedict and Mattson, Timothy G and Ginsburg, Dan},
        year={2011},
        publisher={Pearson Education}
    }

\end{filecontents*}

% \newpage
\bibliographystyle{plain}
\bibliography{bibliography}

% that's all folks
\end{document}
